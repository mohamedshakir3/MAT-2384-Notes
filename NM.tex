\documentclass[openany]{report}
\usepackage[utf8]{inputenc}

\usepackage{stylesheets}
\usepackage{lecture_notes_styles}
\usepackage{pgfplots}
\pgfplotsset{compat=1.18}

\newcommand{\powerset}[0]{\mathcal{P}}

\title{MAT 2384: Numerical Methods Lecture Notes}
\author{Last Updated:}

\begin{document}

\maketitle

\tableofcontents

\chapter{Iterative Methods to Solve The Equation $f(x) = 0$}
Given a continous function $f$, the goal of this chapter is to estimate the solution of the equation $f(x) = 0$ in a certain interval $I$ numerically. 
\begin{theorem}[Intermediate Value Theorem]
    Let $f: [a,b] \rightarrow \real$ be a continous function. Let $y \in \real$ be any value between $f(a)$ and $f(b)$. Then there exists $z \in [a,b]$ such that $f(z) = y$.
\end{theorem}
Bolzano's Theorem is a special case of the Intermediate Value Theorem, which states
\begin{theorem}[Bolzano's Theorem]
    If a continuous function defined on an interval $I$ is sometimes positive and sometimes negative, then it must be 0 at some point. So there exists $x_0 \in I$ such that $f(x_0) = 0$.
\end{theorem}

\begin{proof}
    Without loss of generality, assume $f(a) \leq f(b)$. Let $y \in [f(a), f(b)]$. Set 
    \[S \coloneqq \{x \in [a,b]: f(x) \leq y_0\}\]
    $S$ is a subset of $[a,b]$ so it is bounded, $a \in S$ since $f(a) \leq y_0$. Therefore $S \neq \emptyset$. Thus by completeness, there exists $x_0 \coloneqq \sup S \in [a,b]$. We want $f(x_0) = y_0$. Consider the cases where $f(x_0) = y_0$, $f(x_0) < y_0$, and $f(x_0) \geq y_0$.
    \begin{itemize}
        \item \textbf{Case 1: $f(x_0) = y_0$} This case is trivial since this is the result we want. 
        \item \textbf{Case 2: $f(x_0) < y_0$} Set $\epsilon \coloneqq y_0 - f(x_0)$. Since $f$ is continous at $x_0$, $\exists \delta > 0$ such that 
        \[|x - x_0| < \delta \implies |f(x) - f(x_0)| < \epsilon\]
        Since $f(x_0) < y_0 \leq f(b)$, we can find $x > x_0$ such that $x \in [a,b]$ and $|x-x_0| < \delta$. Then $f(x) < f(x_0) + \epsilon = y_0$. So $x \in S$ by the definition of $S$, but $x > x_0$ which contradicts the fact that $x_0 = \sup S$.
        \item \textbf{Case 3: $f(x_0) > y_0$} Set $\epsilon \coloneqq f(x_0) - y_0$. Since $f$ is continous at $x_0$, $\exists \delta > 0$ such that if $x \in [a,b]$ and $|x-x_0| < \delta$, then $|f(x) - f(x_0)| < \epsilon$. So $f(x) > f(x_0) - \epsilon = y_0$ and $x_0 > a$. We can assume that $x - \delta > a$ since $\delta$ can be arbitrarly small, and we claim $x_0 - \delta$ is an upper bound for $S$. To prove this, if $x > x - \delta$, then either $|x - x_0| < \delta$, in which case $f(x) > f(x_0) - \epsilon = y_0$, or $x > x_0$ then $x \neq S$ since $x_0$ is an upper bound for $S$. Therefore, if $x >  x_0 - \delta$, then $x \neq S$, thus proving the claim. This contradicts that $x_0$ is the supremum of $S$.  
    \end{itemize}
\end{proof}
\noindent
\textbf{Example.} Prove that the equation 
\[2x^3 + 2x - 4 = 0\]
has a unique root in $[0,1]$.
\begin{proof}
    Set $f(x) \coloneqq 3x^2 + 2x - 4$, this function is continuous since it is a polynomial.We have $f(0) = -4 < 0$ and $f(1) = 1 > 0$, so by the intermediate value theorem, there exists $c \in [0,1]$ such that $f(c) = 0$. It follows that $c$ is unique since the polynomial is injective by virtue of $x^3$ and $x$ being injective. 
\end{proof}
\section{Fixed-Point Iteration}
\begin{definition}
    We say that the value $x = r$ is a \emph{fixed point} for a function $g(x)$ if $g(r) = r$.
\end{definition}
\noindent
\textbf{Example.} $g(x) = \frac{5-x^2}{4}$. $r = 1$ is a fixed-point for $g$ since $g(1) = 1$.\\[2ex]
Graphically, fixed-point of $g(x)$ correspond to the intersection of the graph of $g(x)$ and the line $y = x$. Given an equation $f(x) = 0$, we can write it under the form 
\[g(x) = x\]
by isolating one $x$ in the equation.\\[2ex]
\textbf{Example.} $3x^3 + 2x - 5 = 0$. We can write this as 
\[x = \frac{5-3x^3}{2}\]
Set $g(x) \coloneqq \frac{5-3x^3}{2}$. Then $g(x) = x$. Finding a root for $f(x) = 0$ is equivalent to finding a fixed-point for $g(x)$.\\[2ex]
\subsection{Steps to Solving Using Fixed-Point Iteration}
 Start with a first estimation $x_0$ (will be given) of the root, and form the following sequence (known as the \emph{iteration sequence})
    \[x_0, x_1 = g(x_0), x_2 = g(x_1), \dots, x_n = g(x_{n-1})\]
If this sequence converges to a value $a$, then we can prove that $a$ is a fixed-point for $g$, hence a root for $f(x) = 0$.
\begin{theorem}
    Assume that the function $g$ has a fixed-point $s$ on an interval $I$, if 
    \begin{enumerate}[label=(\roman*)]
        \item $g(x)$ is continuous on $I$
        \item $g'(x)$ is continuous on $I$
        \item $|g'(x)| < 1$ for all $x \in I$
    \end{enumerate}
    Then the iteration sequence converges.
\end{theorem}
\noindent
The steps for solving are as follows 
\begin{enumerate}
    \item Start with $f(x) = 0$
    \item Rewrite $f(x) = 0$ under the form $x = g(x)$
    \item Verify that the sequence $x_0, x_1 = g(x_0), x_2 = g(x_1), \dots, x_n = g(x_{n-1})$ converges using the above theorem (or otherwise)
    \item Compute terms of the above sequence and stop when you reach the required accuracy
\end{enumerate}
\textbf{Example.} Consider the equation 
\[x^3 + 12x - 3 = 0\]
\begin{enumerate}
    \item Prove that the equation has a unique root in $[-1.9, 1.9]$
    \item Use the Fixed-Point iteration method to estimate the value of the root to 6 decimal points starting with $x_0 = 1.8$
\end{enumerate}

\textbf{Solution:} Using the steps, we have 
\begin{enumerate}
    \item Set $f(x) \coloneqq x^3 + 12x - 3$. Since $f(x)$ is a polynomial, it is continuous, so by the intermediate value theorem, we have there exists $c \in [-1.9, 1.9]$ such that $f(c) = 0$. $f(x)$ is injective since $x^3$ and $x$ are injective, so $c$ is unique.
    \item Set $g(x) \coloneqq \frac{3-x^3}{12}$.
    \item Checking the conditions of the theorem, $g(x)$ is continuous since it is a polynomial, $g'(x) = -\frac{x^2}{4}$ is continuous since it is a polynomial. Then 
    \[|g'(x)| = \frac{x^2}{4} \leq \frac{1.9^2}{4} = 0.902 < 1\]
    Therefore, the sequence converges. 
    \item We have to calculate the terms of the iteration sequence, 
    \begin{align*}
        &x_0 = 1.8\\
        &x_1 = g(x_0) = \frac{3-1.8^2}{12} = -0.236000\\
        &x_2 = g(x_1) = \frac{3 - (0.236)^2}{12} = 0.251095\\
        &x_3 = g(x_2) = \frac{3 - (0.251095)^2}{12} = 0.24861\\
        &x_4 = g(x_3) = \frac{3 - (0.24861)^2}{12} = 0.248718\\
        &x_5 = g(x_4) = \frac{3 - (0.248718)^2}{12} = 0.248718\\
    \end{align*}
    We stop when 2 consecutive terms agree on the first 6 decimal points. So the root is $0.248718$ correct to 6 decimal points.
\end{enumerate}
\section{Newton's Method}
Newton's method is a technique for solving equations of the form $f(x) = 0$ by successive approximation. The idea is to pick an initial guess $x_0$ such that $f(x_0)$ is reasonably close to 0. We then find the equation of the line tangent to $y = f(x)$ at $x = x_0$, and determine where this tangent line intersects the $x$ axis at the new point $x_1$. So, 
\[x_1 = x_0 - \frac{f(x_0)}{f'(x_0)}\]
We then find the equation of the line tangent to $y = f(x)$ at $x = x_1$, and repeat this process, so we have 
\[x_{n+1} = x_n - \frac{f(x_n)}{f'(x_n)}\]
\textbf{Example.} Using Newton's method, estimate the value of the root of the equation 
\[x^3 + 12x - 3 = 0\]
on $[0,2]$. Start by showing that the equation has a unique root on $[0,2]$, then approximate (to 6 decimal places) with the starting point $x_0 = 1.8$.\\[2ex]
\textbf{Solution.} We have $f(x) = x^3 + 12x - 3$, and 
\[f(0) = -3 \text{ and } f(2) = 29\]
Therefore by the intermediate value theorem, there exists $c \in [0,2]$ such that $f(c) = 0$. $f(x)$ is injective since $f'(x) = 2x^2 + 12$ is striclty increasing on $[0,2]$, so $c$ is unique. Now using Newton's method, 
\[x_0 = 1.8\]
\[x_1 = x_0 - \frac{f(x_0)}{f'(x_0)} = 0.675138\]
\[x_2 = x_1 - \frac{f(x_1)}{f'(x_1)} = 0.270469\]
\[x_3 = x_2 - \frac{f(x_2)}{f'(x_2)} = 0.248748\]
\[x_4 = x_3 - \frac{f(x_3)}{f'(x_3)} = 0.248718\]
\[x_5 = x_4 - \frac{f(x_4)}{f'(x_4)} = 0.248718\]
Therefore, the our root is $0.248718$ correct to 6 decimal places.


\end{document}