
\documentclass{article}
\usepackage[landscape]{geometry}
\usepackage{url}
\usepackage{multicol}
\usepackage{amsmath}
\usepackage{esint}
\usepackage{amsfonts}
\usepackage{tikz}
\usetikzlibrary{decorations.pathmorphing}
\usepackage{amsmath,amssymb}
\usepackage{listings}
\usepackage{colortbl}
\usepackage{xcolor}
\usepackage{mathtools}
\usepackage{amsmath,amssymb}
\usepackage{enumitem}
\usepackage{environ}
\makeatletter

\newcommand*\bigcdot{\mathpalette\bigcdot@{.5}}
\newcommand*\bigcdot@[2]{\mathbin{\vcenter{\hbox{\scalebox{#2}{$\m@th#1\bullet$}}}}}
\makeatother

\title{CSI 2132 Midterm Cheat Sheet}
\usepackage[brazilian]{babel}
\usepackage[utf8]{inputenc}

\advance\topmargin-.8in
\advance\textheight3in
\advance\textwidth3in
\advance\oddsidemargin-1.5in
\advance\evensidemargin-1.5in
\parindent0pt
\parskip2pt
\newcommand{\hr}{\centerline{\rule{3.5in}{1pt}}}
%\colorbox[HTML]{e4e4e4}{\makebox[\textwidth-2\fboxsep][l]{texto}


\definecolor{blue}{HTML}{A7BED3}
\definecolor{brown}{HTML}{DAB894}
\definecolor{pink}{HTML}{FFCAAF}


\newtheorem{theorem}{Theorem}[section]
\newtheorem{definition}{Definition}[section]
\newtheorem{fact}{Fact}[section]
\newtheorem{prop}{Proposition}[section]
\newtheorem{corollary}{Corollary}[section]





\tikzset{header/.style={path picture={
\fill[green, even odd rule, rounded corners]
(path picture bounding box.south west) rectangle (path picture bounding box.north east) 
([shift={( 2pt, 4pt)}] path picture bounding box.south west) -- 
([shift={( 2pt,-2pt)}] path picture bounding box.north west) -- 
([shift={(-2pt,-4pt)}] path picture bounding box.north east) -- 
([shift={(-6pt, 6pt)}] path picture bounding box.south east) -- cycle;
},
label={[anchor=west, fill=green]north west:\textbf{#1:}},
}} 

\tikzstyle{mybox} = [draw=black, fill=white, very thick,
    rectangle, rounded corners, inner sep=10pt, inner ysep=10pt]
\tikzstyle{fancytitle} =[fill=black, text=white, rounded corners, font=\bfseries]


\tikzstyle{bluebox} = [draw=blue, fill=white, very thick,
    rectangle, rounded corners, inner sep=10pt, inner ysep=10pt]
\tikzstyle{bluetitle} =[fill=blue, inner sep=4pt, text=white, font=\small]


\tikzstyle{brownbox} = [draw=brown, fill=white, very thick,
    rectangle, rounded corners, inner sep=10pt, inner ysep=10pt]
\tikzstyle{browntitle} =[fill=brown, inner sep=4pt, text=white, font=\small]

\tikzstyle{pinkbox} = [draw=pink, fill=white, very thick,
    rectangle, rounded corners, inner sep=10pt, inner ysep=10pt]
\tikzstyle{pinktitle} =[fill=pink, inner sep=4pt, text=white, font=\small]

\tikzstyle{redbox} = [draw=red!35, fill=white, very thick,
    rectangle, rounded corners, inner sep=10pt, inner ysep=10pt]
\tikzstyle{redtitle} =[fill=red!35, inner sep=4pt, text=white, font=\small]


\NewEnviron{brownbox}[1]{
    \begin{tikzpicture}
    \node[brownbox](box){%
    \begin{minipage}{0.9\textwidth}
    \BODY
    \end{minipage}};
    \node[browntitle, right=10pt] at (box.north west) {#1};
    \end{tikzpicture}
}

 \NewEnviron{redbox}[1]{
    \begin{tikzpicture}
    \node[redbox](box){%
    \begin{minipage}{0.9\textwidth}
    \BODY
    \end{minipage}};
    \node[redtitle, right=10pt] at (box.north west) {#1};
    \end{tikzpicture}
}
   
    

\NewEnviron{bluebox}[1]{%
\begin{tikzpicture}
    \node[bluebox](box){%
        \begin{minipage}{0.9\textwidth}
            \BODY
        \end{minipage}
    };
    
\node[bluetitle, right=10pt] at (box.north west) {#1};
\end{tikzpicture}
}

\NewEnviron{pinkbox}[1]{%
\begin{tikzpicture}
    \node[pinkbox](box){%
        \begin{minipage}{0.9\textwidth}
            \BODY
        \end{minipage}
    };
    
\node[pinktitle, right=10pt] at (box.north west) {#1};
\end{tikzpicture}
}



\NewEnviron{blackbox}[1]{%
\begin{tikzpicture}
    \node[mybox](box){%
        \begin{minipage}{0.3\textwidth}
        \raggedright
        \small{
            \BODY
        }
        \end{minipage}
    };
    
\node[fancytitle, right=10pt] at (box.north west) {#1};
\end{tikzpicture}
}


\begin{document}

\begin{center}{\large{\textbf{MAT 2384 Midterm 2 Cheat Sheet}}}\\
\end{center}




\begin{multicols*}{3}
\begin{blackbox}{Homogeneous ODEs}
    {\footnotesize
    \begin{bluebox}{Constant Coefficients}
        \textbf{General Form}\\[-2ex]
        \[a_ny^{(n)} + a_{n-1}y^{(n-1)} + \cdots + a_1y' + a_0y = 0\]
        \textbf{Characteristic Equation}
        \[\lambda^n + a_{n-1}\lambda^{(n-1)} + \cdots + a_1\lambda + a_0 = 0\]
        If $\lambda$ is a root with multiplicity $k$, 
        \[y_1 = e^{\lambda x}, \ y_2 = xe^{\lambda x}, \ y_3 = x^2e^{\lambda x}, \ldots, y_k = x^{k-1}e^{\lambda x}\]
        If $\alpha + i\beta$ is a pair of complex conjugate roots, then it contributes the following 2 equations to our basis of solutions,\\[-2ex]
        \[y_1 = e^{\alpha x}\cos(\beta x), \ y_2 = e^{\alpha x}\sin(\beta x)\]
    \end{bluebox}
    \begin{brownbox}{Euler-Cauchy}
        \textbf{General Form}\\[-2ex]
        \[a_n(x)y^{(n)} + a_{n-1}(x)y^{(n-1)} + \cdots + a_1(x)y' + a_0(x)y = 0\]
        \textbf{Characteristic Equation.} Differentiate $y = x^m$ and plug into ODE. If $m$ is a root of the characteristic equation of multiplicity $k$, then it contributes the following equations to our basis of solutions
        \[y_1 = x^m, y_2 = x^m\ln x, y_3 = x^m (\ln x)^2, y_k = x^m(\ln x)^{k-1}\]
        If $\alpha \pm i\beta$ is a pair of complex conjugate roots of the characteristic equation, then the pair contributes the following 2 equations to our basis of solutions
        \[y_1 = x^{\alpha} \cos(\beta \ln x), y_2 = x^\alpha\sin(\beta \ln x)\]
    \end{brownbox}\\[-2ex]
    }
\end{blackbox}
\begin{blackbox}{Undetermined Coefficients}
    \textbf{General Form}\\[-2ex]
    \[a_ny^{(n)} + \cdots + a_1(x)y' + a_0y = r(x)\]
    All coefficients on the left ($a_n, \ldots, a_0$) are constants and $r(x)$ is a polynomial, exponential, and/or sinusoidal.
    \begin{pinkbox}{Rules}
        \begin{itemize}[leftmargin=5pt]
            \item \textbf{Rule 1: Basic Rule.} If $r(x) = ke^{\lambda x}$, then $y_p = Ae^{\lambda x}$. If $r(x) = p_n(x)$, then $y_p = q_n(x)$. If $r(x) = k\cos(wx)$ or $k\sin(wx)$, then $y_p = Ae^{\alpha x}\cos(wx) + Be^{\alpha x}\sin(wx)$. If $r(x) = p_n(x)e^{\lambda x}$, then $y_p = q_n(x)e^{\lambda x}$.
            \item \textbf{Rule 2: The Modifcation Rule.} If any term of $y_p$ is in the basis of solutions for $y_H$, then multiply by $x$ until it's not.
            \item \textbf{Rule 3: The Sum Rule.} If $r(x) = r_1(x) + \cdots r_n(x)$, then do each $r_i(x)$ seperately and combine.
        \end{itemize}

    \end{pinkbox}\\[-2ex]
\end{blackbox}
\begin{blackbox}{Variation of Parameters}
    {\footnotesize
    \textbf{General Form}\\[-2ex]
        \[a_n(x)y^{(n)} + \cdots + a_1(x)y' + a_0y(x) = r(x)\]
        Same solution with $y = y_H + y_p$ where $y_H$ is the solution to the coresponing homogeneous ODE\\[-1.5ex]
        \[a_n(x)y^{(n)} + \cdots + a_1(x)y' + a_0y(x) = 0\]
        \[y_p = u_1y_1 + u_2y_2 + \cdots u_ny_n\]
        Where $\{y_1,y_2,\ldots,y_n\}$ is a basis of solutions for the corresponding homogeneous ODE. $u_1, u_2, \ldots, u_n$ are functions that satisfy the following system of equations 
        \[\begin{cases}
            0 = u_1'y_1 + u2'y_2 + \cdots + u_n'y_n \\
            0 = u_1'y_1' + u2'y_2' + \cdots + u_n'y_n' \\
            \vdots\\
            \frac{r(x)}{a_n(x)} = u_1'y_1^{(n-1)} + u_2'y_2^{(n-1)} + \cdots + u_n'y_n ^{(n-1)} 
        \end{cases}\]
    }
\end{blackbox}
\begin{blackbox}{Systems of ODE's}
    {\footnotesize
    \textbf{General Form}\\[-2ex]
    \[\begin{rcases}
        y_1' = a_{11}y_1 + a_{12}y_2 + r_1(x)\\
        y_1' = a_{21}y_1 + a_{22}y_2 + r_2(x)  
    \end{rcases} \implies \vec{y'} = A\vec{y} + \vec{r}(x)\]
    \[A = \begin{bmatrix}
        a_{11} & a_{12}\\
        a_{21} & a_{22}
    \end{bmatrix}, \vec{y'} = \begin{bmatrix}
        y_1'\\
        y_2'
    \end{bmatrix}, \vec{r}(x) = \begin{bmatrix}
        r_1(x)\\
        r_2(x)
    \end{bmatrix}\]
    \begin{redbox}{Homogenous Systems with Constant Coefficients}
        A system is homogeneous if $\vec{r}(x) = 0$, $\vec{y'} = A\vec{y}$.
        \begin{brownbox}{Steps to Solving}
            \begin{enumerate}[leftmargin=5pt]
                \item Find the eigen values of $A$ $|\lambda I - A| = 0$
                \item If 2 distinct real eigenvalues, find eigenvectors $V_1,V_2$.\\[-5ex]
                \[\vec{y} = c_1\vec{V_1}e^{\lambda_1x} + c_2\vec{V_2}e^{\lambda_2x}\]
                \item If $\lambda$ with multiplicity 2, find generalized eigenvector $\rho$ \\[-4ex]
                \[Y = c_1Ve^{\lambda x} + c_2(xV + \rho)e^{\lambda x}\]                
                \item If $\lambda = \alpha \pm i\beta$, then find eigenvector for $\lambda_1 = \alpha + i\beta$, compute gen. solution $y = c_1V_1 + c_2V_2$ with
                \[\vec{V}e^{(\alpha + i\beta)x} = \vec{V}e^{\alpha x}(\cos(\beta x) + i\sin (\beta x)) = \vec{V_1} + i\vec{V_2}\]
        \end{enumerate}
        \end{brownbox}
    \end{redbox}\\[-2ex]
    \begin{bluebox}{Non-Homogeneous Systems}
        Similar to non-homogeneous ODEs, use undetermined to solve $\vec{y} = \vec{y_H} + \vec{y_p}$. $y_H$ is ODE with $\vec{r}(x) = 0$. Decompose $\vec{r}(x)$ as 
        {\scriptsize
        \[\vec{r(x)} = \begin{bmatrix}
            2 x^3 + x^2 + x \\
            3e^x + x^2 + 2x + 1
        \end{bmatrix} = \begin{bmatrix}
            2\\
            0
        \end{bmatrix}x^3 + \begin{bmatrix}
            0\\
            3
        \end{bmatrix}e^x + \begin{bmatrix}
            1\\
            1
        \end{bmatrix}x^2 + \begin{bmatrix}
            1\\
            2
        \end{bmatrix}x\]
        }
        Solve each $r_1(x)$ using same rules as undetermined coefficients replacing constants with constant vectors
        \[r_1(x) = \begin{bmatrix}
            2\\0
        \end{bmatrix}x^3 \implies \vec{y}_p = \vec{U}x^3 + \vec{V}x^2 + \vec{W}x + \vec{Z} \]
    \end{bluebox}\\[-2ex]
    }
\end{blackbox}
\begin{blackbox}{Newton's Divided-Difference Interpolation}
    {\footnotesize
    Given a node $x_i$, 
    \begin{enumerate}[leftmargin=5pt]
        \item The \emph{first divided difference} at $x_i$ is defined as 
        \[f(x_i,x_{i+1}) = \frac{f_{i+1}- f_i}{x_{i+1} - x_i}\]
        \item The \emph{second divided difference} at $x_i$ is 
        \[f(x_i, x_{i+1}, x_{i+2}) = \frac{f(x_{i+1}, x_{i+2}) - f(x_i,x_{i+1})}{x_{i+2}-x_i}\]
        \item In general, the $k$th divided difference at $x_i$ is 
        \[f(x_i, x_{i+1}, \ldots x_{i+k}) = \frac{f(x_{i+1},\ldots, x_{i+k}) - f(x_i, \ldots x_{i+k-1})}{x_{i+k}-x_i}\]
    \end{enumerate}
    \textbf{Newton's Interpolation Polynomial} 
    \begin{align*}
        p_n(x) &= f_0 + f(x_0,x_1)(x-x_0)\\
        &+ f(x_0,x_1,x_2)(x-x_0)(x-x_1) + \cdots\\
        &+ f(x_0,\ldots,x_n)(x-x_0)(x-x_1)\cdots(x - x_n)
    \end{align*}
    }
\end{blackbox}
\begin{blackbox}{Numerical Integration}
    {\footnotesize
        \begin{bluebox}{Midpoint Rule}
            \[\int_a^b f(x)dx \approx h[f(x_1^*) + f(x_2^*) + \cdots + f(x_n^*)]\]
            \[h = \frac{b-a}{n}, x_i^* = \frac{x_i + x_{i+1}}{2}\]
            Error formula with $M$ being $|f''(x)| \leq M$ for $x \in [a,b]$
            \[E_M \leq \frac{M(b-a)^3}{24n^2}\]
        \end{bluebox}
        \begin{brownbox}{Trapezoidal}
            \[\int_a^b f(x)dx\approx \frac{h}{2}\left[f(a) + 2f(x_1) + 2f(x_2) + \cdots + f(x_n)\right]\]
            \textbf{Error Formula}\\[-2ex]
            \[|E_T| \leq \frac{M(b-a)^3}{12n^2}\]
        \end{brownbox}
        \begin{pinkbox}{Simpsons Rule}
            Divide $[a,b]$ into an \emph{EVEN} number of subintervals\\[-2ex]
            \[\int_a^b f(x) \approx \frac{h}{3}\left[f(a) + 4(x_1) + 2f(x_2) + 4f(x_3) + \cdots + f(b)\right]\]
            \[|E_S| \leq \frac{M(b-a)^5}{180n^4}\]
            \textbf{Error Formula} Where  $M$ is the upperbound for the fourth derivative of $f(x)$\\[-2ex]
            \[|E_S| \leq \frac{M(b-a)^5}{180n^4}\]
        \end{pinkbox}\\[-2ex]
    }
\end{blackbox}
\begin{blackbox}{Gaussian Quadrature}
    To convert $\int_a^bf(x)dx$ into the form $\int_{-1}^1g(t)dt$, use the substitution\\[-4ex]
    \[x = \frac{b-a}{2}t + \frac{b+a}{2}\]
    Then the Gaussian Quadrature formula is\\[-2ex]
    \[\int_{-1}^{1}f(t)dt \approx w_1f(t_1) + \cdots w_nf(t_n)\]
    \begin{redbox}{Table of Nodes and Coefficients}
        \begin{center}
            \begin{tabular}{c|cc}
                Order $n$ & Nodes $t_i$ & Coefficients $w_i$\\
                \hline
                1 & 0 & 2\\
                \hline
                2 & $-0.5773502692$ & 1\\
                & $0.5773502692$ & 1\\
                \hline
                & $-0.7745966692$ & $0.555555556$\\
                3 & 0 & 0.888888889\\
                & $0.7745966692$ & $0.555555556$\\
                \hline
                & -0.8611363116 &0.3478548451\\
                4&-0.3399810436& 0.6521451549\\
                &0.3399810436 &0.6521451549\\
                &0.8611363116 &0.3478548451\\
                \hline
                &-0.9061798459 &0.2369268850\\
                &-0.5384693101& 0.4786286705\\
                5 &0.0 &0.5688888889\\
                &0.5384693101 &0.4786286705\\
                &0.9061798459& 0.2369268850\\
            \end{tabular}
        \end{center}
    \end{redbox}\\[-2ex]
\end{blackbox}
\begin{blackbox}{Linear Algebra and Integrals}
    \begin{pinkbox}{Linear Algebra}      
        \textbf{Eigen Value}\\[-2ex]
        \[|A - \lambda I| = |\lambda I - A| = 0\]
        \[A = \begin{bmatrix}
            a & b\\
            c & d
        \end{bmatrix} \implies A^{-1} = \frac{1}{\det A}\begin{bmatrix}
            d & -b\\
            -c & a
        \end{bmatrix}\]
        \textbf{Eigen Vectors.} An eigenvector $\vec{V}$ is a vector that satisfies\\[-4ex]
        \[[A - \lambda I | \vec{0}]\]
        \textbf{Generalized Eigen Vector $\rho$} The solution to, then pick specific value for the parameter $t$ to get a specific vector, (i.e take $t=0$, $t=1$, etc).\\[-2ex]
        \[[A - \lambda I | V]\]
    \end{pinkbox}
    \begin{brownbox}{Integrals}
        \[\int xe^{ax} = \frac{1}{a}xe^{ax} - \frac{1}{a^2}e^{ax}, \int \ln x = x\ln x - x \]
        \[\int \sin(ax)dx = -\frac{1}{a}\cos(ax), \int \cos(ax)dx = \frac{1}{a}\sin (ax)\]
    \end{brownbox}\\[-2ex]
\end{blackbox}
\begin{blackbox}{Example of Non-Homogeneous System}
    {\footnotesize
        \textbf{Example.}\\[-2ex] 
        \[\vec{y'} = \begin{bmatrix}
            9 & 18\\
            -2 & -3
        \end{bmatrix}\vec{y} + \begin{bmatrix}
            9x - 51\\
            7 + e^{2x}
        \end{bmatrix}; \ \vec{y}(0) = \begin{bmatrix}
            1\\
            0
        \end{bmatrix}\]
        \textbf{Solution.} Solve the corresponding homogeneous ODE
        \[\vec{y}' = \begin{bmatrix}
            9 & 18\\
            -2 & -3
        \end{bmatrix}\vec{y}\]
        Find eigenvalues of $A$
        \[\det\begin{bmatrix}
            9 - \lambda & 18\\
            -2 & -3 - \lambda\\
        \end{bmatrix} = (9 - \lambda)(-3-\lambda) + 36 = \lambda^2 -6\lambda + 9\]
        This gives us $\lambda = 3$ with multiplicity 2, find eigenvector $V = t\begin{bmatrix}-3\\1\end{bmatrix}$
        Find generalized eigenvector, set parameter $t = 0$,
        \[[A - 3I| V] = \begin{bmatrix}
            6 & 18 & -3\\
            -2 & -6 & 1
        \end{bmatrix} \sim \begin{bmatrix}
            1 & 3 & -1/2\\
            0 & 0 & 0
        \end{bmatrix}\]
        \[\vec{\rho} = \begin{bmatrix}
            -3t - 1/2\\
            t
        \end{bmatrix} \implies \begin{bmatrix}
            -1/2\\
            0
        \end{bmatrix}\]
        The general solution to the homogeneous ODE is 
        \begin{align*}
            \vec{y}_H &= c_1\begin{bmatrix}-3\\1\end{bmatrix}xe^{3x} + c_2\left(x\begin{bmatrix}
                -3\\
                1
            \end{bmatrix} + \begin{bmatrix}
                -1/2\\
                0
            \end{bmatrix}\right)e^{3x}  \\
            &= \begin{bmatrix}
                -3c_1e^{3x} - 3c_2xe^{3x} - \frac{1}{2}c_2e^{3x}\\
                c_1e^{3x} + c_2xe^{3x}
            \end{bmatrix}
        \end{align*}
        For $y_p$, decompose $r(x)$ to get
        \[r(x) = \begin{bmatrix}
            9x - 51\\
            7 + e^{2x}
        \end{bmatrix} = \begin{bmatrix}
            9\\
            0
        \end{bmatrix}x + \begin{bmatrix}
            -51\\
            7
        \end{bmatrix} + \begin{bmatrix}
            0\\
            1
        \end{bmatrix}e^{2x}\]
        Find $y_p$ for each part of $r(x)$\\[-2ex]
        \[y_p = Ux + V + We^{2x}\]
        Rewrite the non-homogeneous system as 
        \[\vec{y'} = \begin{bmatrix}
            9 & 18\\
            -2 & -3
        \end{bmatrix}y + \begin{bmatrix}
            9\\0
        \end{bmatrix}x + \begin{bmatrix}
            -51\\7
        \end{bmatrix} + \begin{bmatrix}
            0\\1
        \end{bmatrix}e^{2x}\]
       Compute $\vec{y'}_p$ and plug into the system to solve for constant vectors, $\vec{y'}_p = U + 2We^{2x}$, $\vec{y'}_p = A\vec{y}_p + \vec{r}(x)$\\[-2ex]
       {\scriptsize
       \[
        U + 2We^{2x} = A(Ux + V + We^{2x}) + \begin{bmatrix}
            9\\0
        \end{bmatrix}x + \begin{bmatrix}
            -51\\7
        \end{bmatrix} + \begin{bmatrix}
            0\\1
        \end{bmatrix}e^{2x}
        \]
    }
        This gives us the three equations 
        \[AU + \begin{bmatrix}
            9\\0
        \end{bmatrix} = 0, AV + \begin{bmatrix}
            -51\\7
        \end{bmatrix} = U, AW + \begin{bmatrix}
            0\\1
        \end{bmatrix} = 2W\]
        Solve equations to find\\[-2ex]
       \[U = \begin{bmatrix}
        3\\-2
       \end{bmatrix}, V = \begin{bmatrix}
        0\\3
       \end{bmatrix}, W = \begin{bmatrix}
        18\\-7
       \end{bmatrix}\]
        This gives us our particular solution
        \[
            y_p = \begin{bmatrix}
                3\\-2
            \end{bmatrix}x + \begin{bmatrix}
                0\\3
            \end{bmatrix} + \begin{bmatrix}
                18\\-7
            \end{bmatrix}e^{2x} = \begin{bmatrix}
                3x + 18e^{2x}\\
                -2x+3 - 7e^{2x}
            \end{bmatrix}  
        \]
        The general solution to the non-homogeneous ODE is 
        \[
            y = \begin{bmatrix}
                -3c_1e^{3x} - 3c_2xe^{3x} - \frac{1}{2}e^{3x}\\
                c_1e^{3x} + c_2xe^{3x}
            \end{bmatrix}
            + \begin{bmatrix}
                3x + 18e^{2x}\\
                -2x+3 - 7e^{2x}
            \end{bmatrix}
        \]
    }
\end{blackbox}
\begin{blackbox}{Extra Notes}
    \hspace{300pt}
    \vspace{557pt}
\end{blackbox}
\end{multicols*}

\end{document}
