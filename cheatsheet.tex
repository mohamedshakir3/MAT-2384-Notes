
\documentclass{article}
\usepackage[landscape]{geometry}
\usepackage{url}
\usepackage{multicol}
\usepackage{amsmath}
\usepackage{esint}
\usepackage{amsfonts}
\usepackage{tikz}
\usetikzlibrary{decorations.pathmorphing}
\usepackage{amsmath,amssymb}
\usepackage{listings}
\usepackage{colortbl}
\usepackage{xcolor}
\usepackage{mathtools}
\usepackage{amsmath,amssymb}
\usepackage{enumitem}
\usepackage{environ}
\makeatletter

\newcommand*\bigcdot{\mathpalette\bigcdot@{.5}}
\newcommand*\bigcdot@[2]{\mathbin{\vcenter{\hbox{\scalebox{#2}{$\m@th#1\bullet$}}}}}
\makeatother

\title{CSI 2132 Midterm Cheat Sheet}
\usepackage[brazilian]{babel}
\usepackage[utf8]{inputenc}

\advance\topmargin-.8in
\advance\textheight3in
\advance\textwidth3in
\advance\oddsidemargin-1.5in
\advance\evensidemargin-1.5in
\parindent0pt
\parskip2pt
\newcommand{\hr}{\centerline{\rule{3.5in}{1pt}}}
%\colorbox[HTML]{e4e4e4}{\makebox[\textwidth-2\fboxsep][l]{texto}


\definecolor{blue}{HTML}{A7BED3}
\definecolor{brown}{HTML}{DAB894}
\definecolor{pink}{HTML}{FFCAAF}


\newtheorem{theorem}{Theorem}[section]
\newtheorem{definition}{Definition}[section]
\newtheorem{fact}{Fact}[section]
\newtheorem{prop}{Proposition}[section]
\newtheorem{corollary}{Corollary}[section]





\tikzset{header/.style={path picture={
\fill[green, even odd rule, rounded corners]
(path picture bounding box.south west) rectangle (path picture bounding box.north east) 
([shift={( 2pt, 4pt)}] path picture bounding box.south west) -- 
([shift={( 2pt,-2pt)}] path picture bounding box.north west) -- 
([shift={(-2pt,-4pt)}] path picture bounding box.north east) -- 
([shift={(-6pt, 6pt)}] path picture bounding box.south east) -- cycle;
},
label={[anchor=west, fill=green]north west:\textbf{#1:}},
}} 

\tikzstyle{mybox} = [draw=black, fill=white, very thick,
    rectangle, rounded corners, inner sep=10pt, inner ysep=10pt]
\tikzstyle{fancytitle} =[fill=black, text=white, rounded corners, font=\bfseries]


\tikzstyle{bluebox} = [draw=blue, fill=white, very thick,
    rectangle, rounded corners, inner sep=10pt, inner ysep=10pt]
\tikzstyle{bluetitle} =[fill=blue, inner sep=4pt, text=white, font=\small]


\tikzstyle{brownbox} = [draw=brown, fill=white, very thick,
    rectangle, rounded corners, inner sep=10pt, inner ysep=10pt]
\tikzstyle{browntitle} =[fill=brown, inner sep=4pt, text=white, font=\small]

\tikzstyle{pinkbox} = [draw=pink, fill=white, very thick,
    rectangle, rounded corners, inner sep=10pt, inner ysep=10pt]
\tikzstyle{pinktitle} =[fill=pink, inner sep=4pt, text=white, font=\small]

\tikzstyle{redbox} = [draw=red!35, fill=white, very thick,
    rectangle, rounded corners, inner sep=10pt, inner ysep=10pt]
\tikzstyle{redtitle} =[fill=red!35, inner sep=4pt, text=white, font=\small]


\NewEnviron{brownbox}[1]{
    \begin{tikzpicture}
    \node[brownbox](box){%
    \begin{minipage}{0.9\textwidth}
    \BODY
    \end{minipage}};
    \node[browntitle, right=10pt] at (box.north west) {#1};
    \end{tikzpicture}
}

 \NewEnviron{redbox}[1]{
    \begin{tikzpicture}
    \node[redbox](box){%
    \begin{minipage}{0.9\textwidth}
    \BODY
    \end{minipage}};
    \node[redtitle, right=10pt] at (box.north west) {#1};
    \end{tikzpicture}
}
   
    

\NewEnviron{bluebox}[1]{%
\begin{tikzpicture}
    \node[bluebox](box){%
        \begin{minipage}{0.9\textwidth}
            \BODY
        \end{minipage}
    };
    
\node[bluetitle, right=10pt] at (box.north west) {#1};
\end{tikzpicture}
}

\NewEnviron{pinkbox}[1]{%
\begin{tikzpicture}
    \node[pinkbox](box){%
        \begin{minipage}{0.9\textwidth}
            \BODY
        \end{minipage}
    };
    
\node[pinktitle, right=10pt] at (box.north west) {#1};
\end{tikzpicture}
}



\NewEnviron{blackbox}[1]{%
\begin{tikzpicture}
    \node[mybox](box){%
        \begin{minipage}{0.3\textwidth}
        \raggedright
        \small{
            \BODY
        }
        \end{minipage}
    };
    
\node[fancytitle, right=10pt] at (box.north west) {#1};
\end{tikzpicture}
}


\begin{document}

\begin{center}{\large{\textbf{MAT 2384 Summary Sheet: Ordinary Differential Equations}}}\\
\end{center}




\begin{multicols*}{3}
\begin{blackbox}{Seperable First Order ODE's}
    {\scriptsize
    
    \textbf{Definition.} A first order ODE is called seperable if it can be written in the form \\[-2ex]
    \[F(x)dx = G(y)dy\]
        \begin{redbox}{Steps to Solving Seperable ODE's}
            \begin{enumerate}[align=left]
                \item Write $y' = \frac{dy}{dx}$
                \item Seperate the ODE to write it in the form $F(x)dx = G(y)dy$
                \item Integrate both sides
                \item If an initial condition is given, solve for the integration constant C. 
            \end{enumerate}
        \end{redbox}\\[-2ex]
    }
\end{blackbox}
\begin{blackbox}{Homogeneous Coefficients}
    {\scriptsize
    \textbf{Definition.} A function $F(x,y)$ is called homogeneous of degree $k$ if \\[-2ex]
    \[F(\lambda x, \lambda y) = \lambda^k \cdot F(x,y)\]
    \textbf{Definition.} A first order ODE given in differential form is called homogeneous if both $M(x,y)$ and $N(x,y)$ are homogeneous of the same degree.\\[1ex]
    \textbf{Theorem.} A first order ODE of homogeneous coefficients can be made seperable by changing the function by substituting $u = \frac{y}{x} \implies y = xu$ or $u \frac{x}{y} \implies y = \frac{x}{u}$. 
    }
\end{blackbox}
\begin{blackbox}{Exact First Order ODE's}
    {\scriptsize
    
    \textbf{Definition.} Given a function $F(x,y)$, the differential of $F$ denoted by $dF$ is defined by\\[-2ex]
    \[dF = \frac{\partial F}{\partial x} dx + \frac{\partial F}{\partial y}dy\]
    \textbf{Remark.} $dF = 0 \iff F(x,y) = C$.\\[1ex]
    \textbf{Definition.} A first order ODE is called exact if there exists $F(x,y)$ such that 
    \[\frac{\partial F}{\partial x} = M(x,y) \text{ and } \frac{\partial F}{\partial x} = N(x,y)\]
    \textbf{Theorem.} A first order ODE is exact if and only if
    \[\frac{\partial M}{\partial y} = \frac{\partial N}{\partial x}\]
    \begin{bluebox}{Steps To Solving}
        \begin{enumerate}[align=left]
            \item Check exactness: $\frac{\partial M}{\partial y} = \frac{\partial N}{\partial x}$
            \item Look for a function $F(x,y)$ such that\\[-0.0001mm]
            \[\frac{\partial F}{\partial x} = M, \ \frac{\partial F}{\partial y} = N\]
            Do this by integrating $M$ with respect to $x$ or $N$ with respect to $y$ then differentiate the equation with respect to the other variable respectively.
            \item The general solution is $F(x,y) = C$
            \item If an initial condition is given, solve for the integration constant $C$.
        \end{enumerate}
    \end{bluebox}\\[-2ex]
    }
\end{blackbox} 
\begin{blackbox}{ODEs With an Integrating Factor}
    {\scriptsize
    \textbf{Definition.} We say that the function $\mu(x,y)$ is an integrating factor of a first order ODE in differential form if the following equation is exact \\[-2ex]
    \[\mu(x,y)M(x,y)dx + \mu(x,y)N(x,y)dy = 0\]
    \begin{redbox}{Theorem}
        If \\[-6ex]
        \[\frac{\frac{\partial M}{\partial y} - \frac{\partial N}{\partial x}}{M} = g(y)\]
        for some function $g(y)$, then\\[-2ex]
        \[\mu(y) = \exp\left(-\int g(y) dy\right)\]
        If \\[-5ex]
        \[\frac{\frac{\partial M}{\partial y} - \frac{\partial N}{\partial x}}{M} = f(x)\]
        for some function $f(x)$, then\\[-2ex]
        \[\mu(x) = \exp\left(\int f(x) dy\right)\]
    \end{redbox}\\[-2ex]
    }
\end{blackbox}
\begin{blackbox}{Linear First-Order ODEs}
    {\scriptsize
        \textbf{Definition.} A first order ODE is called linear if it can be written in the form\\[-2ex]
        \[y' + f(x)y = r(x)\]
        \begin{bluebox}{Steps to Finding General Solution}
            Given a linear first-order ODE in the form $y' + f(x)y = r(x)$, find $y$ using 
            \[y = \left(\int \exp\left(\int f(x)dx\right)r(x)dx + C\right)\exp\left(\int f(x)\right)^{-1}\]
        \end{bluebox}\\[-2ex]
    }
\end{blackbox} 
\begin{blackbox}{Bernouilli First Order ODE's}
    {\scriptsize
            \textbf{Definition.} A first-order ODE is called of \emph{Bernouilli} type if it can be written in the form for some $a \in \mathbb{R}$.  \\[-2ex]
            \[y' + f(x)y = r(x)y^a\]
        \begin{redbox}{Steps to Solving Bernoulli type ODE's}
            \begin{enumerate}
                \item Let $u = y^{1-a}$
                \item Compute $u'$: $u' = (1-a)y^{-1}y'$
                \item Isolate $y'$ from the original ODE and substitute into $u'$
                \item The resulting ODE is linear that we solve for $u$.  
            \end{enumerate}
        \end{redbox}\\[-2ex]
    }
\end{blackbox}
\begin{blackbox}{Fixed-Point Iteration}
    {\scriptsize 
    \textbf{Theorem.} The iteration sequence $x_0, x_1 = g(x_0), \ldots$ converges if $g$ is continuous and $g'$ is continuous and $|g(x)| < 1$. 
    \begin{bluebox}{Steps to Solving}
        Start with $f(x) = 0$, rewrite $f(x) = 0$ as $x = g(x)$. Verify the sequence $x_0, x_1 = g(x_0), \ldots$ converges using the above theorem. Compute the terms of the sequence until 2 terms agree on the required number of decimal places.
    \end{bluebox}\\[-2ex]
    }
\end{blackbox}
\begin{blackbox}{Newton's Method}
    {\scriptsize
    Given an equation equation $f(x) = 0$ and a starting point $x_0$, the Newton's method is given as\\[-2ex]
    \[x_{n+1} = x_n - \frac{f(x_n)}{f'(x_n)}\]
    Calculate values for $x_n$ until you reach the accuracy.\\
    \textbf{Example.} Approximate the root for the equation $x^3 + 12x - 3$ to 6 decimal places with $x_0 = 1.8$  \\
    \textbf{Solution.}\\[-5ex]
    \[x_1 = x_0 - \frac{f(x_0)}{f'(x_0)} = 0.675138, \ldots, x_3 = x_2 - \frac{f(x_2)}{f'(x_2)} = 0.248748\]\\[-4ex]        \[x_4 = x_3 - \frac{f(x_3)}{f'(x_3)} = 0.248718, \ x_5 = x_4 - \frac{f(x_4)}{f'(x_4)} = 0.248718\]
    }
\end{blackbox}
\begin{blackbox}
    
\end{blackbox}

\end{multicols*}
\begin{center}{\large{\textbf{Numerical Methods}}}\\
\end{center}
\begin{multicols*}{3}
    \begin{blackbox}{Iterative Methods to Solving $f(x) = 0$}
        \textbf{Theorem.} (Intermediate Value Theorem). Let $f: [a,b] \rightarrow \mathbb{R}$ be a continuous function. Let $y \in \mathbb{R}$ between $f(a)$ and $f(b)$. Then there exists $z \in [a,b]$ such that $f(z) = y$.
        \begin{bluebox}{Fixed Point Iteration}
            \textbf{Definition.} The value $x = r$ is a fixed point for $g(x)$ if $g(r) = r$.\\
            \textbf{Definition.} The iteration sequence is given as \\[-2ex]
            \[x_{n+1} = g(x_n)\]
            with $x_0$ being given. \\[1ex]
            \textbf{Theorem.} Assume that the function $g(x)$ has a fixed-point $s$ on an interal $I$, if 
            \begin{itemize}
                \item $g(x)$ is continuous on $I$,
                \item $g'(x)$ is continuous on $I$, and
                \item $|g'(x)| < 1$ for all $x \in I$.
            \end{itemize}
            Then then the iteration sequence converges.
            \begin{redbox}{Steps to Solve Using Fixed-Point Iteration}
               
                Then the steps for solving are as follows,
                \begin{enumerate}
                    \item Start with $f(x) = 0$
                    \item Rewrite $f(x) = 0$ under the form $x = g(x)$
                    \item Verify the iteration sequence $x_0, x_1 = g(x_0), \ldots, x_n = g(x_{n-1})$ converges using the above theorem 
                    \item Compute the terms of the sequence and stop when the required accuracy is reached (i.e when 2 consecutive terms have the same $k$ decimal places where $k$ is the desired accuracy).
                \end{enumerate}
            \end{redbox}
        \end{bluebox}\\[-2ex]
    \end{blackbox}
    \begin{blackbox}{Newton's Method}
        Given an equation equation $f(x) = 0$ and a starting point $x_0$, the Newton's method is given as\\[-2ex]
        \[x_{n+1} = x_n - \frac{f(x_n)}{f'(x_n)}\]
        Calculate values for $x_n$ until you reach the accuracy.\\
        {\footnotesize
        \textbf{Example.} Approximate the root for the equation $x^3 + 12x - 3$ to 6 decimal places with $x_0 = 1.8$  \\
        \textbf{Solution.}\\[-5ex]
        \[x_1 = x_0 - \frac{f(x_0)}{f'(x_0)} = 0.675138, \ldots, x_3 = x_2 - \frac{f(x_2)}{f'(x_2)} = 0.248748\]\\[-4ex]        \[x_4 = x_3 - \frac{f(x_3)}{f'(x_3)} = 0.248718, \ x_5 = x_4 - \frac{f(x_4)}{f'(x_4)} = 0.248718\]
        }
    \end{blackbox}
\end{multicols*}
\end{document}
