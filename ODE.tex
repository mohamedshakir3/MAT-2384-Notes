\documentclass[openany]{report}
\usepackage[utf8]{inputenc}

\usepackage{stylesheets}
\usepackage{lecture_notes_styles}
\usepackage{pgfplots}
\pgfplotsset{compat=1.18}

\newcommand{\powerset}[0]{\mathcal{P}}

\title{MAT 2384: Ordinary Differentials Lecture Notes}
\author{Last Updated:}

\begin{document}

\maketitle

\tableofcontents
\setcounter{chapter}{-1}
\chapter{Introduction and Basic Terminology}
\begin{definition}[Differential Equations]
    A differential equation is an equation involving an unknown function $y$ (of one or many variables), derivatives of $y$, and other known functions of indepedent variables.
\end{definition}
\begin{definition}[Order of Differential Equations]
    The \emph{order} of a differential equation is the highest order of a derivative appearing in the equation.
\end{definition}
If the unknown function $y$ is a function of only one variable, $y = f(x)$, we saw that the differential equation is \emph{ordinary}. If $y$ is a function of two or more variables, we say the differential equation is a \emph{partial} differential equation. \\[2ex]
\textbf{Example.}
\[x^3y'' - 3e^x \sin x y' + 3y = \tan x\]
This is an ODE of order 2. \\[2ex]
\textbf{Example.}
\[x_1x_2 \frac{\partial^2y}{\partial x_1\partial x_2} - 3e^{x_1} \frac{\partial y}{\partial x_1} = 0\]
This is a PDE of order 2. \\[2ex]
\begin{center}
    \textbf{Note:} In this course, we will only consider ODEs.
\end{center}
\begin{definition}
    We say that the function $y$ is a \emph{solution} to a differential equation on an interval $I$ if $y$ is well-defined on $I$ and $y$ satisfies the differential equation.
\end{definition}
\textbf{Example.} Consider the differential equation 
\[y'' - 5y' + 4y = 0\]
Show that the function 
\[y = Ae^x + Be^{4x}\]
is a solution for the differential equation on $\real$ for any constants $A$ and $B$. \\[2ex]
\textbf{Solution:} We have $y = Ae^x + Be^{4x}$ is well defined on $\real$. 
\[y' = Ae^x - 4Be^{4x}\]
\[y'' = Ae^x + 16Be^{4x}\]
So, 
    \[y'' - 5y' + 4y = Ae^x + 16Be^{4x} - 5Ae^x - 20Be^{4x} + 4Ae^x + 4Be^{4x} = 0\]
Therefore, $y = Ae^x + Be^{4x}$ is a solution to the differential equation for any $A,B \in \real$. This is called the \emph{general solution} to the differential equation. \\[2ex]
\textbf{Remark:} The above example shows that a differential equation has infinitely many solutions.
\begin{definition}[Initial Value Problem]
    An \emph{intial value problem} (IVP) of order n consists of an ordinary differential equation of order $n$, and $n$ initial coniditions of the form 
    \[y(x_0) = y_0, \ \ y'(x_0) = y_1 \ldots\]
    \[y^{(n-1)}(x_0) = y_{n-1}\]
\end{definition}
\begin{center}
    \textbf{Note:} $y^{(i)}$ denotes the $i$th derivative of $y$.
\end{center}
\textbf{Example.} Consider the IVP of order 3 
\[y''' - 3e^xy'' + 6xy' + 2y = x^2\]
\[y(0) = -1 \ \ y'(0) = 2 \ \ y''(0) = 1\]
\textbf{Example.} Solve the following IVP
\[y'' - 5y + 4y = 0\]
\[y(0) = 1 \ \ y'(0) = 2\]
\textbf{Solution:} We saw in the previous example that the general solution to this differential equation is 
\[y = Ae^x + Be^{4x}\]
We can use the initial conditions to find the constants $A$ and $B$.
\[y(0) = 1 \implies 1 = Ae^0 + Be^0 = A + B\]
\[y'(0) = 2 \implies 2 = Ae^0 - 4Be^0 = A + 4B\]
\[A + 4B - A - B = 2 - 1 \implies 3B = 1 \implies B = \frac{1}{3} \ \ A = \frac{2}{3}\]
\begin{theorem}[Existence and Uniqueness Theorem for the First Order ODEs]
    Consider the IVP: 
    \[y' = F(x,y), \ \ y(x_0) = y_0\]
    \begin{itemize}
        \item \textbf{Existence:} If $F(x,y)$ is continuous in an open rectangular region 
        \[R = \left\{(x,y) \in \real^2: a < x < b, c < y < d\right\}\]
        of the $xy$-plane that contains the initial point $(x_0, y_0)$, then there exists a solution $y(x)$ to the intial value problem that is defined in some open interval $I = (\alpha, \beta)$ containg $x_0$.
        \item \textbf{Uniqueness:} If the partial derivative $\frac{\partial F}{\partial y}$ of the function $F(x,y)$ is continuous in the recnagular region $R$, then the solution $y(x)$ is unique.
    \end{itemize}
\end{theorem}
\begin{center}
    \textbf{Note:} We will always suppose this condition is satisfied in this course.
\end{center}

\chapter{Ordinary Differential Equations of First Order}
The goal of this chapter is to solve ODE's of order 1. 
\begin{definition}
    The \emph{standard form} of an ODE of order 1 is an expression of the form 
    \[y' = f(x,y)\]
    We can rewrite $y'$ as $\frac{dy}{dx}$ and we have the \emph{differential form} 
    \[M(x,y)dx + N(x,y)dy = 0\]
\end{definition}
\noindent
\textbf{Example.} Consider the differential equation 
\[2xy' + 3y = 2y' + \sin x\]
The standard form is
\[2xy ' - 2y' = \sin x - 3y \implies y' = \frac{\sin x - 3y}{2x-2}\]
The differential form is 
\begin{align*}
    &2x\frac{dy}{dx} + 3y = 2\frac{dy}{dx} + \sin x   \\
    \implies&2xdy + 3ydx = 2dy + \sin x dx\\
    \implies& (3y - \sin x)dx (2x-2)dy = 0
\end{align*}
\section{Seperable First Order Ordinary Differential Equations}

\begin{definition}
    A first order ODE is called \emph{seperable} if it can be written in the form
    \[F(x)dx = G(y)dy\]
\end{definition}
\subsection{Solving Seperable ODE's} 
To solve a seperable ODE,
\begin{enumerate}
    \item Write $y' = \frac{dy}{dx}$
    \item Seperate the ODE to write it in the form 
    \[F(x)dx = G(y)dy\]
    \item Take integrals of both sides
    \item If an initial condition is given, solve for the constant of integration $C$. 
\end{enumerate}
\textbf{Example.} Solve the IVP 
\[(y^2 + 1)y' = \frac{x}{y} \ \ y(1) = 1\]
\textbf{Solution:} We can write $y' = \frac{dy}{dx}$ and we get 
    \[(y^2 + 1)\frac{dy}{dx} = \frac{x}{y} \implies (y^2 + 1)ydy = xdx\]
    Taking integrals on both sides, we have 
    \[\int y^3 + ydy = \int xdx \implies \frac{y^4}{4} + \frac{y^2}{2} = \frac{x^2}{2} + C\]
    Using our initial condition, we have $y = 1$ when $x =1$, then 
    \[\frac{1}{4} + \frac{1}{2} = \frac{1}{2} + C\]
    Therefore $C = \frac{1}{2}$ and the solution to the IVP is
    \[\frac{y^4}{4} + \frac{y^2}{2} = \frac{x}{2} + \frac{1}{4}\]
    This is called the \emph{implicit solution} since we could not explcitly solve for $y$ in terms of $x$. \\[2ex]

\textbf{Example.} Solve the IVP
\[e^xy' = (x+1)y^2 \ \ y(0) = -\frac{1}{2}\]
\textbf{Solution:} 
\begin{align*}
    &e^x\frac{dy}{dx} = (x+1)y^2 \\
    \implies& \frac{1}{y^2}dy = \frac{x+1}{e^x}dx \\
    \implies& \int \frac{1}{y^2}dy = \int (x+1)e^{-x}dx \\
\end{align*}
We can use integration by parts to solve the right hand side integral. Let $u = x+1$ and $dv = e^{-x}dx$, $u' = 1$, and $v = -e^{-x}$. Then 
\begin{align*}
    \int (x+1)e^{-x}dx &= uv - \int u'vdx\\
    &= -(x+1)e^{-x} - \int -e^{-x}dx\\
    &= -(x+1)e^{-x} - e^{-x} + C\\
\end{align*}
Therefore we have 
\begin{align*}
    \frac{y^{-2 +1}}{-2 + 1} &= -(x+1)e^{-x} - e^{-x} + C\\
    -\frac{1}{y} &= -(x+1)e^{-x} - e^{-x} + C\\
\end{align*}
Setting $y = -\frac{1}{2}$ and $x = 0$, we have 
\[2 = -2 + C \implies C = 4\]
Therefore the implicit solution is 
\[-\frac{1}{y} = -(x+1)e^{-x} - e^{-x} - 4\]
We can rewrite this as an explicit solution as 
\[y = \frac{1}{(x+2)e^{-x}-4}\]
\section{First Order ODE's With Homogeneous Coefficients}
\begin{definition}
    A function $F(x,y)$ of two variables is called \emph{homogeneous} of degree $k$ if 
    \[F(\lambda x, \lambda y) = \lambda^k \cdot F(x,y)\]
\end{definition}
This type of ODEs can be made seperable after a suitable change of variables of the unknown function.\\[2ex]
\noindent
\textbf{Example.}
\[F(x,y) = 3x^2y - 2xy^2 + y^3\]
We can check if its homogeneous by the definition, 
\begin{align*}
    F(\lambda x, \lambda y) &= 3(\lambda x)^2 (\lambda y) - 2(\lambda x) (\lambda y)^2 + (\lambda y^3)\\
    &= 3\lambda^3x^2y - 2\lambda^3xy^2 + \lambda^3y^3\\
    &= \lambda^3(3x^2y - 2xy^2 + y^3)\\
    &= \lambda^3F(x,y)
\end{align*}
Therefore, $F(x,y)$ is homogeneous of degree 3. We can tell quickly if a polynomial is homogeneous is by looking at the exponents of each term. If the sum of the exponents of each term is the same, then the polynomial is homogeneous, with order being the sum of the exponents in each term (i.e $x^2y$ has exponents 2,1, $xy^2$ has exponents 1,2, and $y^3$ has exponents 3, each sum to 3).\\[2ex]
\begin{definition}
    A first order ODE given in differential form
    \[M(x,y)dx + N(x,y)dy = 0\]
    is called of \emph{homogeneous coefficients} if both $M(x,y)$ and $N(x,y)$ are homogeneous of the same degree.
\end{definition}
\noindent
\textbf{Example.}
\[(3x^2+2y^2+2xy)dx - 4xydy = 0\]
Both terms are homogeneous of degree 2, therefore this is a differential equation of homogeneous coefficients. 
\begin{theorem}
    A first order ODE of homogeneous coefficients can be made seperable by changing the function using one of the following substitutions:
    \begin{itemize}
        \item Set $u \coloneqq \frac{y}{x}$ or
        \item $u \coloneqq \frac{x}{y}$
    \end{itemize}
\end{theorem}
\noindent
\textbf{Example.} Solve the following IVP 
\[(x^2-y^2)dx + 2xydy = 0 \ \ y(1) = 2\]
\textbf{Solution:} This is a first order ODE with homogeneous coefficients. Let 
\[u \coloneqq \frac{y}{x} \implies y = xu\]
\[\frac{dy}{dx} = 1 \cdot u + x \cdot \frac{du}{dx} \implies dy = udx + xdu\]
So, we have 
\[(x^2-y^2)dx + 2xydy = 0 \implies (x^2 - x^2u^2)dx + 2x(xu)(udx + xdu) = 0\]
Simplyfing, we get 
\begin{align*}
    x^2dx - x^2u^2dx + 2x^2u^2dx + 2x^3udu &= 0\\
    dx - u^2dx + 2u^2dx + 2xudu &= 0\\
    (1 + u^2)dx + 2xudu &= 0\\
    (1+u^2)dx &= -2xudu\\
    -\frac{1}{x}dx &= \frac{2u}{1+u^2}du
\end{align*}
Now that it's seperable, we can integrate both sides,
\begin{align*}
    -\int \frac{1}{x}dx &= \int \frac{2u}{1+u^2}du\\
    -\ln(x) &= \ln(1+u^2) + C\\
\end{align*}
Now using our initial condition, we have $y(1) = 2$. But, our differential equation is a function of $u$ not $y$, so we must calculate $u$ using our initial condition. So, $u(1) = \frac{y(1)}{1} = 2$. So, 
\[-\ln 1 = \ln 5 + C \implies C = -\ln 5\]
Therefore, our solution is
\begin{align*}
    \ln x &= \ln(1 + u^2) - \ln 5\\
    \ln\left(\frac{5}{x}\right) &= \ln(1 + u^2)\\
    \frac{5}{x} &= 1 + u^2\\
    u^2 &= \frac{5}{x} - 1\\
    \frac{y^2}{x^2} &= \frac{5}{x} - 1\\
    y &= \sqrt{5x - x^2}
\end{align*}
We take the positive square root since if we took the negative square root, then $y(1) = -2$ which is not our initial condition.\\[2ex]
\textbf{Example.} Solve the IVP 
\[(2x + y)dx - xdy = 0 \ \ y(1) = -2 \ \ x > 0\]
\textbf{Solution:} This is a first order ODE with homogeneous coefficients. Let
\[u = \frac{y}{x} \implies y = xu\]
\[dy = udx + xdu\]
Substituting into our differential equation, we get
\begin{align*}
    (2x + xu)dx - x(udx + xdu) &= 0\\
    (2+u)dx - (udx + xdu) &= 0\\  
    2dx + udx - udx - xdu &= 0\\
    2dx &= xdu \implies \frac{2}{x}dx = du
\end{align*}
This differential equation in $u$ is sperable, so we can integrate 
\begin{align*}
    \int \frac{2}{x}dx &= \int du\\
    2\ln x &= u + C\\
\end{align*}
Using your initial condition, $y(1) = -2$. so $u(1) = \frac{y(1)}{1} = -2$. Therefore,
\[2\ln 1 = -2 + c \implies C = 2\]
Now solving for $y$, 
\begin{align*}
    u &= 2\ln x - 2\\
    \frac{y}{x} &= 2\ln x - 2\\
    y &= x(2\ln x - 2)
\end{align*}
This is our explicit solution to the initial value problem. 
\section{Exact First Order ODEs}
\begin{definition}
    Given a function $F(x,y)$ of two variables, the differential of $F(x,y)$ denoted by $dF$ is defined by 
    \[dF = \frac{\partial F}{\partial x} dx + \frac{\partial F}{\partial y}dy\]
\end{definition}
\noindent
\textbf{Example.} Let 
\[F(x,y) = 2x^2y^3 + \sin(x+2y)\]
Then 
\[dF = (4xy^3 + \cos(x+2y))dx + (6x^2y^2 + 2\cos(x+2y))dy\]
\textbf{Remark:}
\[dF = 0 \iff \frac{\partial F}{\partial x} dx + \frac{\partial F}{\partial y} dy = 0 \iff \frac{\partial F}{\partial x} = 0 \text{ and } \frac{\partial F}{\partial y} = 0 \]
So, $F(x,y) = C$ is a constant function. Therefore, 
\[dF = 0 \iff F(x,y) = C\]
\begin{definition}
    A first order ODE 
    \[M(x,y)dx + N(x,y)dy = 0\]
    is called \emph{exact} if there exists a continuous function $F(x,y)$ such that 
    \[\frac{\partial F}{\partial x} = M(x,y) \text{ and } \frac{\partial F}{\partial x} = N(x,y)\]
    So if $M(x,y)dx + N(x,y)dy = 0$ is exact, then 
    \[dF = 0 \implies F(x,y) = C\]
\end{definition}
In summary, if $M(x,y)dx + N(x,y)dy = 0$ is exact, then find $F(x,y)$ such that
\[\frac{\partial F}{\partial x} = M(x,y) \text{ and } \frac{\partial F}{\partial x} = N(x,y)\]
Then, the (implicit) solution to the ODE is $F(x,y) = C$. Furthermore, since $M(x,y) = \frac{\partial F}{\partial x}$ and $N(x,y) = \frac{\partial F}{\partial y}$, then 
\[\frac{\partial M}{\partial y} = \frac{\partial}{\partial y} \left(\frac{\partial F}{\partial x}\right) = \frac{\partial^2F}{\partial x \partial y}\]
\[\frac{\partial N}{\partial x} = \frac{\partial}{\partial x} \left(\frac{\partial F}{\partial y}\right) = \frac{\partial^2F}{\partial y \partial x}\]
So by the Clairaut-Schwarz Theorem, the ODE is exact if and only if
\[\frac{\partial M}{\partial y} = \frac{\partial N}{\partial x}\]
\begin{theorem}[Condition for Exactness]
    The first order ODE $M(x,y)dx + N(x,y)dy = 0$ (with $M,N$ continuous) is exact if and only if
    \[\frac{\partial M}{\partial y} = \frac{\partial N}{\partial x}\]
\end{theorem}
\subsection{Steps to Solving Exact ODEs}
\begin{enumerate}
    \item Check exactness: $\frac{\partial M}{\partial y} = \frac{\partial N}{\partial x}$
    \item Look for a function $F(x,y)$ such that
    \[\frac{\partial F}{\partial x} = M \ \ \frac{\partial F}{\partial y} = N\]
    \item The general solution to the ODE is $F(x,y) = C$.
    \item If an intial condition is given, use it to find $C$.
\end{enumerate}
\textbf{Example.} Solve the following IVP
\[(6x - 2y^2 + 2xy^3)dx + (3x^2y^2 - 4xy)dy = 0, \ \ y(1) = -2\]
\textbf{Solution.} We first check exactness.
\[\frac{\partial M}{\partial y} = -4y + 6xy^2 = 6xy^2 - 4y\]
\[\frac{\partial N}{\partial x} = 6xy^2 - 4y\]
Therefore, this ODE is exact. Now we need to find a function $F(x,y)$ satisfying the partial derivatives. We can do this by integrating $N$ with respect to $y$, so we have 
\[\frac{\partial F}{\partial y} = 3x^2y^2 - 4xy\]
\[F(x,y) = \int 3x^2y^2-4xy dy = 3x^2 \int y^2dy - 4x \int ydy = x^2y^3 - 2xy^2 + h(x)\]
We add $h(x)$ since when integrating with respect to $y$, we are treating $x$ as a constant so $h(x)$ is constant with respect to $y$. So we have 
\[F(x,y) = x^2y^3 - 2xy^2 + h(x)\]
Now we can use the first equation to solve for $h(x)$, 
\[\frac{\partial F}{\partial x} = 2xy^3 - 2y^2 + h'(x)\]
This equation is equal to $M$, so we can plug $M$ in and get 
\[M = 6x-2y^2 + 2xy^3 = 2xy^3 - 2y^2 + h'(x) \implies h'(x) = 6x\]
Now we can solve for $h(x)$ by taking the integral, 
\[h(x) = \int 6xdx = 3x^2 + C_1\]
Now, we get 
\[F(x,y) = x^2y^3 - 2xy^2 + 3x^2 + C_1\]
So the general solution to the ODE is
\[x^2y^3 - 2xy^2 + 3x^2 + C_1 = C_2 \implies x^2y^3 - 2xy^2 + 3x^3 = C\]
Now using the intial condition $y(1) = - 2$, then 
\[1^2 (-2)^3 - 2(1)(-2)^2 + 3(1)^2 = C \implies C  = -13\]
Therefore, the solution to the IVP is 
\[x^2y^3 - 2xy^2 + 3x^3 = -13\]
\textbf{Example.} Solve the IVP 
\[(2x\cos(y) - 3x^2y + ye^{xy})dx + (-x^2\sin(y)+xe^{xy} - x^3)dy = 0, \ \ y(0) = 1 \]
\textbf{Solution.} We first check exactness,
\[\frac{\partial M}{\partial y} = -2x\sin(y) - 3x^2 + e^{xy} + ye^{xy}\]
\[\frac{\partial N}{\partial x} = -2x\sin(y) + ye^{xy} + e^{xy} -3x^2\]
Therefore this ODE is exact, so we look for $F(x,y)$ such that 
\[\frac{\partial F}{\partial x} = M \text{ and } \frac{\partial F}{\partial y} = N\]
\begin{align*}
    F(x,y) &= \int 2x\cos(y) - 3x^2y + ye^{xy}dx\\
    &= 2\cos(y)\int xdx - 3y \int x^2dx + y\int e^{xy}dx\\
    &=x^2 - x^3y + y\frac{e^{xy}}{y} + h(y)
\end{align*}
So we have
\[F(x,y) = x^2\cos(y) - x^3y + e^{xy} + h(y)\]
Then, 
\[\frac{\partial F}{\partial y} = -x^2\sin(y) - x^3 + xe^{xy} + h'(y) = N \implies h'(y) = 0\]
So $h(y)$ is a constant, say $h(y) = K$, then our general solution for $F(x,y)$ is 
\[F(x,y) = x^2\cos(y) - x^3y + e^{xy} + k \implies x^2\cos(y) - x^3y + e^{xy} = C\]
Using the condition, $y(0) = 1$, we get
\[(0)^2\cos(1) - (0)^3(1) + e^{0\cdot 1} = C \implies C = 1 \]
Therefore the (implicit) solution to the IVP is
\[x^2\cos(y) - x^3y + e^{xy} = 1\]
\section{First Order ODEs With an Integrating Factor}

\begin{definition}[Integrating Factor]
    We say that the function $\mu(x,y)$ is an \emph{integrating factor} of the first-order ODE 
    \[M(x,y)dx + N(x,y)dy = 0\]
    if the new ODE
    \[\mu(x,y)M(x,y)dx + \mu(x,y)N(x,y)dy = 0\]
    is exact.
\end{definition}
In general, finding an integrating factor is not easy. However, there are some special cases where we can find an integrating factor easily.
\begin{theorem}
    For the ODE
    \[M(x,y)dx + N(x,y)dy = 0\]
    \begin{enumerate}
        \item If 
        \[\frac{\frac{\partial M}{\partial y} - \frac{\partial N}{\partial x}}{M} = g(y)\]
        for some function $g$ of $y$ only, then an integration factor exists given by 
        \[\mu(y) = \exp \left(-\int g(y)dy\right)\]
        \item If
        \[\frac{\frac{\partial M}{\partial y} - \frac{\partial N}{\partial x}}{M} = f(x)\]
        for some function $f$ of $x$ only, then an integration factor exists given by
        \[\mu(x) = \exp\left(\int f(x)dx\right)\]
    \end{enumerate}
\end{theorem}
\textbf{Example.} Solve the IVP 
\[(y^4 + xy)dx + (xy^3 - x^2 + 2y^3e^y)dy = 0, \ \ y(0) = 1\]
\textbf{Solution.} It's clear this ODE is not exact, so we need to find an integrating factor. 
\[\frac{\partial M}{\partial y} - \frac{\partial N}{\partial x} = 4y^3 + x - y^3 + 2x = 3y^3 + 3x\]
If we dvidie by $M$, we get 
\[\frac{\frac{\partial M}{\partial y} - \frac{\partial N}{\partial x}}{M} = \frac{3(y^3 + x)}{y^4 + xy} = \frac{3(y^3 + x)}{y(y^3 + x)} = \frac{3}{y}\]
Therefore, we have our integrating factor 
\[\mu(y) = \exp \left(-\int \frac{3}{y}dy\right) = \exp\left(-3\int\frac{1}{y}dy\right) = \exp\left(\ln(y^{-3})\right) = y^{-3}\]
We multiply the original ODE with $\mu(y) = y^{-3}$
\[y^{-3}(y^4 + xy)dx + y^{-3}(xy^3 - x^2 + 2y^3e^y)dy = (y + xy^2)dx + (x - x^2y^{-3}2e^y)\]
Now we can check the exactness of this ODE, 
\[\frac{\partial M}{\partial y} = 1 - 2xy^{-3} \text{ and } \frac{\partial N}{\partial x} = 1-2xy^{-3}\]
Therefore, this ODE is exact, so we look for $F(x,y)$ such that
\[\frac{\partial F}{\partial x} = M \text{ and } \frac{\partial F}{\partial y} = N\]
The first equation is simpler so we will start with that,
\begin{align*}
    F(x,y) &= \int y + xy^{-2}dx\\
    &= y \int dx + y^{-2}\int xdx\\
    &= xy + \frac{x^2y^{-2}}{2}
\end{align*}
Now we derive with respect to $y$ and use the second equation, 
\[\frac{\partial F}{\partial y} = x - x^2y^{-3} + h'(y) = N = x - x^2y^{-3} + 2e^y \implies h'(y) = 2e^y\]
Then 
\[h(y) = \int h'(y)dy = \int 2e^ydy = 2e^{y} + k\]
So we get the function 
\[F(x,y) = xy + \frac{x^2y^{-2}}{2} + 2e^y + k\]
Then setting $F(x,y)$ equal to a constant to get our (implicit) general solution, 
\[xy+ \frac{x^2y^{-2}}{2} + 2e^y = C\]
Using the initial condition, $y(0) = 1$, we get
\[(0)(1) + \frac{(0)^2(1)^{-2}}{2} + 2e^{1} = C \implies C = 2e\]
Therefore, the (implicit) solution to the IVP is
\[xy+ \frac{x^2y^{-2}}{2} + 2e^y = 2e\]
\textbf{Example.} Solve the following IVP
\[(x^2+4xy + 3y^2)dx + (x^2 + 2xy)dy = 0, \ \ y(1) = 1, \ x > 0\]
\textbf{Solution.} We can see that this function is of homogeneous coefficients, but we will solve it using the integrating factor. First we calculate the partial derivatives,
\[\frac{\partial M}{\partial y} = 4x + 6y\]
\[\frac{\partial N}{\partial x} = 2x + 2y\]
we see that these are not equal so this ODE is not exact, then we calculate the difference,
\[\frac{\partial M}{\partial y} - \frac{\partial N}{\partial x} = 2x + 4y\]
Then to obtain a function of only $x$, we divide by $N$, 
\[\frac{\frac{\partial M}{\partial y} - \frac{\partial N}{\partial x}}{N} = \frac{2x + 4y}{x^2 + 2xy} = \frac{2(x+2y)}{x(x + 2y)} = \frac{2}{x} = f(x)\]
An integrating factor exists and is given by
\[\mu(x) = \exp\left(\int(f(x)dx)\right) = \exp\left(2\ln x\right) = x^2\]
Now we can multiply the original ODE by $\mu(x) = x^2$,
\[(x^2 + 4x^3y + 3x^2y^2)dx + (x^4 + 2x^3y)dy = 0\]
Now we can check the exactness of this ODE, we'll denote the new ODE by $M^*$ and $N^*$,
\[\frac{\partial M^*}{\partial y} = 4x^3 + 6x^2y\]
\[\frac{\partial N^*}{\partial x} = 4x^3 + 6x^2y\]
Therefore, this ODE is exact, so we look for $F(x,y)$ such that
\[\frac{\partial F}{\partial x} = M^* \text{ and } \frac{\partial F}{\partial y} = N^*\]
The second equation is simpler so we'll start with this one, 
\[F(x,y) = \int x^4 + 2x^3ydy = x^4y + x^3y^2 + h(x)\]
Now we derive with respect to $x$ and use the first equation,
\[\frac{\partial F}{\partial x} = 4x^3 + 3x^2y^2 + h'(x)\]
Then, $M^* = x^4 + 4x^3y + 3x^2y^2$, which gives us $h'(x) = x^4$. Now 
\[h(x) = \int x^4dx = \frac{x^5}{5} + k\]
So we get the function
\[F(x,y) = x^4y + x^3y^2 + \frac{x^5}{5} + k\]
The general solution is given by setting $F(x,y)$ equal to a constant,
\[x^4y + x^3y^2 + \frac{x^5}{5} = C\]
Then using our initial value $y(1) = 1$,
\[1 + 1  + \frac{1}{5} = \frac{11}{5}\]
Thus, the solution to the IVP is 
\[x^4y + x^3y^2 + \frac{x^5}{5} = \frac{11}{5}\]
\textbf{Example.} Solve the following IVP with initial condition $y(0) = 1$,
\[(3xy - 2y^2\sin x + 4y)dx + (3x^2 + 8x + 6y \cos x)dy = 0\]
\textbf{Solution.} With $\sin$ and $\cos$ in our function, its certainly not of homogeneous coefficients, we check for exactness, 
\[\frac{\partial M}{\partial y} = 3x - 4y \sin x + 4\]
\[\frac{\partial N}{\partial x} = 6x + 8 - 6y \sin x\]
We see that these are not equal so this ODE is not exact, then we calculate the difference of the partial derivatives 
\[\frac{\partial M}{\partial y} - \frac{\partial N}{\partial x} = 3x + 2y\sin x x  -4\]
Then divide by $M$ to find a function of $y$, 
\[\frac{\frac{\partial M}{\partial y} - \frac{\partial N}{\partial x}}{M} = \frac{-3x + 2y\sin x - 4}{3xy - 2y^2\sin x + 4y} = \frac{-3x + 2y\sin x-4}{-y(-3x+2y\sin x - 4) = -\frac{1}{y}}\]
An integrating factor exists and is given by
\[\mu(y) = \exp\left(-\int -\frac{1}{y}\right) = e^{\ln y} = y\]
Multiplying the original ODE by $\mu(y) = y$,
\[(3xy^2 - 2y^3\sin x + 4y^2)dx + (3x^2y + 8xy + 6y^2\cos x)dy = 0 \]
Now checking exactness of our new ODE, 
\[\frac{\partial M^*}{\partial y} = 6xy - 6y^2 \sin x + 8y\]
\[\frac{\partial N^*}{\partial y} = 6xy - 6y^2 \sin x + 8y\]
This ODE is exact and we can now solve for $F(x,y)$ such that
\[\frac{\partial F}{\partial x} = M^* \text{ and } \frac{\partial F}{\partial y} = N^*\]
Both of the equations are similar in complexity so we'll start with the first one,
\[F(x,y) = \int 3xy^2 - 2y^3 \sin x + 4y dx = \frac{3x^2y^2}{2} = 2y^2\cos x + 4xy^2 + h(y)\]
Then using the second equation to solve for $y$, 
\[\frac{\partial F}{\partial y} = 3x^2y + 6y^2\cos x + 8xy + h'(y)\]
Then, $N^* = 3x^2y + 8xy + 6y^2\cos x$, which gives us $h'(y) = 0$. So we get $h(x) = k$, and the function
\[F(x,y) = \frac{3x^2y^2}{2} + 2y^3 \cos x + 4xy^2 + k\]
The general solution is 
\[\frac{3x^2y^2}{2} + 2y^3 \cos x + 4xy^2 = C\]
Using our inital condition $y(0) = 1$, we get
\[0 + 2 + 0 = C\]
The solution to the IVP is 
\[\frac{3x^2y^2}{2} + 2y^3 \cos x + 4xy^2 = 2\]
\textbf{Example.} Find the general solution of the following ODE,
\[(e^{x+y} ye^y)dx + (xe^y - 1)dy = 0\]
\textbf{Solution.} We have exponential functions so it is certainly not of homogeneous coefficients, we first check for exactness,
\[\frac{\partial M}{\partial y} = e^{x+y} + e^y + ye^y\]
\[\frac{\partial N}{\partial x} = e^y\]
We see that these are not equal so this ODE is not exact, then we calculate the difference of the partial derivatives
\[\frac{\partial M}{\partial y} - \frac{\partial N}{\partial x} = e^{x+y} + ye^y\]
This function is exactly $M$, so we will divide by $M$ to find a function of $y$,
\[\frac{\frac{\partial M}{\partial y} - \frac{\partial N}{\partial x}}{M} = 1 = g(y)\]
An integrating factor exists and is given by
\[\mu(y) = \exp\left(-\int g(y)dy\right) = \exp\left(-\int 1 dy\right) = e^{-y}\]
Multiplying the original ODE by $\mu(y) = e^{-y}$,
\[(e^x + y)dx + (x - e^{-y})dy = 0\]
Now checking exactness of our new ODE,
\[\frac{\partial M^*}{\partial y} = 1\]
\[\frac{\partial N^*}{\partial y} = 1\]
This ODE is exact and we can now solve for $F(x,y)$ such that
\[\frac{\partial F}{\partial x} = M^* \text{ and } \frac{\partial F}{\partial y} = N^*\]
Both of the equations are similar in complexity so we'll start with the first one,
\[F(x,y) = \int e^x + y dx = xe^x + yx + h(y)\]
Then using the second equation to solve for $h(y)$,
\[\frac{\partial F}{\partial y} = x + h'(x)\]
Then, $N^* = x - e^{-y}$, which gives us $h'(y) = -e^{-y}$. So we get
\[h(y) = \int h'(y)dy = -\int e^{-y} = e^y + k\]
Then the general solution is 
\[e^x + xy + e^{-y} = C\]
\section{Linear First-Order ODEs}
\begin{definition}
    A first order ODE that can be written under the form 
    \[y' + f(x)y = r(x)\]
    is called \textbf{linear}.
\end{definition}
\noindent
\textbf{Example.} 
\[xy' + e^xy = \frac{\sin x}{1 + x^2}\]
We can divide by $x$ to get
\[y' + \frac{e^x}{x}y = \frac{\sin x}{x(1+x^2)}\]
This is linear with $f(x) = \frac{e^x}{x}$ and $r(x) = \frac{\sin x}{x(1+x^2)}$
\subsection{Steps to Finding (Explicit) Solutions}
Given a linear first-order ODE in the formm $y' + f(x)y = r(x)$, we can find the solution by following these steps. Start by writing the ODE in differential form by replacing $y'$ with $\frac{dy}{dx}$,
    \begin{align*}
        &\frac{dy}{dx} + f(x)y = r(x)\\
        &\implies dy + f(x)ydx = r(x)dy\\
        &\implies (f(x)y - r(x))dx + 1dy = 0  
    \end{align*}
    This ODE is not exact since 
    \[\frac{\partial M}{\partial y} = f(x)\]
    \[\frac{\partial N}{\partial x} = 1\]
    So we find the integrating factor,
    \[\frac{\frac{\partial M}{\partial y} - \frac{\partial N}{\partial x}}{N} = f(x)\]
    and we get our integrating factor, 
    \[\mu(x) = \exp\left(\int f(x)dx\right)\]
    Note that 
    \[\mu'(x) = \exp\left(f(x)dx\right)\cdot \left(\int f(x)dx\right)' = \mu(x) f(x)\]
    Now we can multply the ODE by $\mu(x)$,
    \[(\mu(x)f(x)y - r(x)\mu(x))dx + \mu(x)dy = 0\]
    Now we can check for exactness, of our new ODE, 
    \[\frac{\partial M^*}{\partial y} =\mu(x)f(x)\]
    \[\frac{\partial n^*}{\partial x} =\mu(x)f(x)\]
    Then we look for our function $F(x,y)$ such that
    \[\frac{\partial F}{\partial x} = M^* \text{ and } \frac{\partial F}{\partial y} = N^*\]
    \[F(x,y) = \int \mu(x) dy = \mu(x)y + h(x)\]
    Then we use the second equation to solve for $h(x)$,
    \[\frac{\partial F}{\partial y} = \mu'(x)y + h'(x) = \mu(x)f(x) + h'(x)\]
    Then, $M^* = \mu(x)f(x)y - \mu(x)r(x)$, which gives us that $h'(x) = - \mu(x)r(x)$. So we get
    \[h(x) = \int \mu(x)r(x)dx  + k \]
    Then we get the function 
    \[F(x,y) = \mu(x)y - \int \mu(x)r(x)dx + k\]
    and our (implicit) general solution is 
    \[\mu(x)y - \int\mu(x)r(x)dx = C\]
    We can find an explicit solution by solving for $y$, 
    \begin{align*}
        &\mu(x)y - \int\mu(x)r(x)dx = C\\
        \implies &\mu(x)y = \int\mu(x)r(x)dx + C\\
        \implies &y = \frac{\int\mu(x)r(x)dx + C}{\mu(x)}\\
        \implies &y = \frac{\int\mu(x)r(x)dx + C}{\exp\left(\int f(x)dx\right)}\\
        \implies &y = \left(\int \exp\left(\int f(x)dx\right)r(x)dx + C\right)\exp\left(\int f(x)\right)^{-1}\\
        \implies &y = \left(\int \exp\left(\int f(x)dx\right)r(x)dx + C\right)\exp\left(-\int f(x)\right)
    \end{align*}
\textbf{Example.} Solve the IVP 
\[y' - 2xy = x \ \ y(0) = \frac{1}{2}\]
\textbf{Solution.} Clearly this is a linear first order ODE witih $f(x) = -2x$, and $r(x) = x$. Then we can apply the formula to get the general solution
\begin{align*}
    y &= \left(\int \exp\left(\int f(x)dx\right)r(x)dx +C \right)\exp\left(-\int f(x)dx\right)  \\
    &=  \left(\int \exp\left(\int -x^2dx\right)xdx + C\right)\exp\left(-\int -2xdx\right)\\
    &= \left(\int x e^{-x^2}xdx + C\right)e^{x^2}\\
\end{align*}
We can solve the integral by substitution, set $u \coloneqq -x^2$, $du = -2xdx$, $dx = \frac{du}{-2x}$, then we get 
\begin{align*}
    \int xe^{-x^2}dx &= \int xe^u \frac{du}{-2x}\\
    & = -\frac{1}{2}\int e^udu\\
    &= -\frac{1}{2} e^{-x^2}
\end{align*}
Then we get the explicit general solution
\[y =  \left(-\frac{1}{2}e^{-x^2} + C\right)  e^{x^2} = -\frac{1}{2} + Ce^{x^2}\]
We can solve for $C$ using the initial condition $y(0) = \frac{1}{2}$,
\[\frac{1}{2} = -\frac{1}{2} + C \implies C = 1\]
So our explicit solution is 
\[y = e^{x^2} -\frac{1}{2}\] 
\textbf{Example.} Solve the IVP
\[y' - 4y = x; \ y(0) = \frac{15}{16}\]
\textbf{Solution.} This is a linear first-order ODE with $f(x) = -4$, and $r(x) = x$. Then we can apply the formula to get the general solution
\begin{align*}
    y &= \left(\int \exp\left(\int f(x)dx\right)r(x)dx + C\right)\exp\left(-\int f(x)\right)\\
    &= \left(\int \exp\left(\int -4dx\right)xdx + C\right)\exp\left(-\int -4dx\right)\\
    &= \left(\int e^{-4x}xdx + C\right)e^{4x}\\
\end{align*}
Now we can solve the integral by parts, set $u = x$, $v' = e^{4x}$, $u' = 1$, $v = \int e^{-4x}dx = -\frac{1}{4}e^{-4x}$, then we get 
\begin{align*}
    \int e^{-4x}xdx &= uv - \int u'vdx\\
    &= -\frac{1}{4}xe^{-4x} - \int -\frac{1}{4}e^{-4x}dx\\
    &= -\frac{1}{4}xe^{-4x} - \frac{1}{16}e^{-4x}
\end{align*}
Then, we get the general solution 
\[y = \left(-\frac{1}{4}xe^{-4x} - \frac{1}{16}e^{-4x} + C\right)e^{4x} = -\frac{1}{4}x - \frac{1}{16} + Ce^{4x}\]
Using our intial condition $y(0) = \frac{15}{16}$, we can solve for $C$,
\[\frac{15}{16} = -\frac{1}{16} + C \implies C = 1 \]
Therefore, our explicit solution is
\[y = -\frac{1}{4}x - \frac{1}{16} + e^{4x}\]
\textbf{Example.} Solve the IVP
\[(1 + \cos x)y' - (\sin x)y = 2x; \ y(0) = \frac{1}{2}\]
\textbf{Solution.} We have to make the coefficient of $y'$ to be 1, so we divide both sides by $1 + \cos x$ to get
\[y' - \frac{\sin x}{1 + \cos x} y = \frac{2x}{1 + \cos x}\]
Now we can use our formula for linear ODE's,
\begin{align*}
    y &= \left(\int \exp\left(\int f(x)dx\right)r(x)dx + C\right)\exp\left(-\int f(x)dx\right)\\
    &= \left(\int \exp\left(\int - \frac{\sin x}{1 + \cos x}dx\right)\frac{2x}{1 + \cos x}dx + C\right)\exp\left(-\int- \frac{\sin x}{1 + \cos x}\right)\\
\end{align*}
Now computing the integral of $f(x)$, set $u = 1 + \cos x$, $du = -\sin x dx$
\begin{align*}
    \int \frac{-\sin x}{1 + \cos x} dx &= \int \frac{-\sin x}{u} \frac{du}{- \sin x}\\
    &= \int \frac{1}{u}du\\
    &= \ln|u| \\
    &= \ln (1 + \cos x)
\end{align*} 
Therefore, we have 
\begin{align*}
    y &= \left(\int \exp\left(\int - \frac{\sin x}{1 + \cos x}dx\right)\frac{2x}{1 + \cos x}dx + C\right)\exp\left(-\int- \frac{\sin x}{1 + \cos x}\right)\\
    &= \left(\int \exp(\ln(1 + \cos x))\frac{2x}{1 + \cos x}dx + C\right)\exp\left(-\ln (1 + \cos x)\right)\\
    &= \left(\int (1 + \cos x)\frac{2x}{1 + \cos x}dx + C\right)(1 + \cos x)^{-1}\\
    &= \frac{\left(\int 2xdx + C\right)}{1 + \cos x}\\
    &= \frac{x^2 + C}{1 + \cos x}\\
\end{align*}
Using our inital condition $y(0) = \frac{1}{2}$, we get 
\[\frac{1}{2} = \frac{C}{2} \implies C = 1\]
Therefore, our explicit solution is
\[y = \frac{x^2 + 1}{1 + \cos x}\]
\section{First-Order Bernoulli ODE's}
\begin{definition}
    A first-order ODE is called of \emph{Bernouilli} type if it can be written in the form
    \[y' + f(x)y = r(x)y^a\]
    for some $a \in \real$.  
\end{definition}
\subsection{Steps to Solving Bernoulli type ODE's}
\begin{enumerate}
    \item Let $u = y^{1-a}$
    \item Compute $u'$:
    \[u' = (1-a)y^{-1}y'\]
    \item Isolate $y'$ from the original ODE and substitute into $u'$
    \item The resulting ODE is linear that we solve for $u$.  
\end{enumerate}
\textbf{Example.} Solve the IVP
\[y' + \frac{4}{x}y = -x^2y^2; \ x > 0, \ y(1) = \frac{1}{3}\]
\textbf{Solution.} This is a first order Bernouilli ODE, with $f(x) = \frac{4}{x}$, $r(x)  = -x^2$, and $a = 2$. Let $u = y^{1 - a} = y^{-1}$, then 
\begin{align*}
    u' &= -y^{-2}y' = -y^{-2}\left(-\frac{4}{x}y - x^3y^2\right)\\
    &= \frac{4}{x}y^{-1} + x^3\\
    &= \frac{4}{x}u + x^3\\
    &= u' - \frac{4}{x}u = x^3
\end{align*}
Now this is a linear first-order ODE in the function $u$ with $f(x) = -\frac{4}{x}$ and $r(x) = x^3$. Then, the general solution is
\begin{align*}
    u &= \left(\int \exp\left(\int f(x)dx\right)r(x)dx + C\right)\exp\left(-\int f(x) \right)\\
    &= \left(\int \exp\left(\int -\frac{4}{x}\right)x^3dx + C\right)\exp\left(-\int -\frac{4}{x}dx\right)\\
    &= \left(\int \exp(-4\ln x)x^3 dx+ C \right)\exp(4\ln x)\\
    &= \left(\int x^{-4}x^3dx + C\right)x^4\\
    &= \left(\int x^{-1}dx + C\right)x^4\\
    &= \left(\ln x + C\right)x^4\\
\end{align*}
Now, we know $u = y^{-1}$, so
\[y = \frac{1}{x^4(\ln x + C)}\]
Using our intial condition $y(1) = \frac{1}{3}$, 
\[\frac{1}{3} = \frac{1}{1 \cdot (\ln 1 + C)} = \frac{1}{C} \implies C = 3\]
Therefore, the explicit solution to our IVP is 
\[y = \frac{1}{x^4(\ln x + 3)}\]
\textbf{Example.} Solve the IVP
\[y' + \frac{2}{x}y = 2 \sqrt{y}; \ x > 0 \ y(1) = 1\]
\textbf{Solution.} This is a first order Bernouilli ODE, with $f(x) = \frac{2}{x}$, $r(x) = 2$, and $a = \frac{1}{2}$. Let $u = y^{1 - a} = y^{\frac{1}{2}} = \sqrt{y}$. Then, 
\begin{align*}
    u' &= \frac{1}{2}y^{-\frac{1}{2}}y'\\
    &= \frac{1}{2}y^{-\frac{1}{2}}\left(-\frac{2}{x}y + 2y^{\frac{1}{2}}\right)\\
    &= -\frac{1}{x}y^{\frac{1}{2}} + 1
\end{align*} 
Then we get the linear first order ODE in $u$
\[u' + \frac{1}{x}u = 1\]
Our general solution for $u$ is 
\begin{align*}
    u &= \left(\int \exp\left(\int f(x)dx\right)r(x)dx + C\right)\exp\left(-\int f(x) dx\right)\\
    &= \left(\int \exp\left(\int\frac{1}{x}dx\right)1dx + C\right)\exp\left(-\int \frac{1}{x}dx\right)\\
    &= \left(\int \exp(\ln x)dx + C\right)\exp(-\ln x)\\
    &= \left(\int xdx + C\right)\exp(-2\ln x)\\
    &= \left(\frac{1}{2}x^2 + C\right)x^{-1}\\
    &= \frac{x}{2} + \frac{C}{x}
\end{align*}
Then using your equation for $u$ in terms of $y$, 
\[u = \sqrt{y} \implies y =\left(\frac{x}{2}+  \frac{C}{x}\right)^2\]
Using our initial condition $y(1) = 1$, 
\[1 = \left(\frac{1}{2} + C\right)^2 \implies C = \frac{1}{2}\]
Therefore, the explicit solution to our IVP is
\[y = \left(\frac{x^2 + 1}{2x}\right)^2\]
\chapter{Second Order Linear Homogeneous ODEs}
\begin{definition}[Linear Independence]
    We say that the functions $y_1, y_2, \ldots, y_n$ are linearly indepedent on an interval $I$ if 
    \[c_1y_1 + c_2y_2 + \cdots + c_ny_n = 0 \implies c_1 = c_2 = \cdots = c_n = 0\]
\end{definition}
\begin{theorem}
    Two functions $y_1,y_2$ are linearly indepedent if and only if $\frac{y_1}{y_2}$ does not equal a constant. 
\end{theorem}
\end{document}